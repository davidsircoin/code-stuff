\documentclass{article}

\usepackage[a4paper, total={6in, 8in}]{geometry} % Page margins
\usepackage[utf8]{inputenc}
\usepackage{amsmath, amssymb, mathtools, amsfonts, amsthm}
\usepackage{wasysym} % Smiley QED
\usepackage{eulervm} % Font
\usepackage{fancyhdr} % Custom headers and footers
\fancyhead[C]{\thepage} % Page numbering for center header
\usepackage{mdframed}
\usepackage{xcolor}
% list environments               
\usepackage{enumerate}            
\usepackage[shortlabels]{enumitem}      

% Code block environment
\usepackage{listings}            
\lstset{basicstyle = \ttfamily, mathescape}


% figure support
\usepackage{import}
\usepackage{xifthen}
\pdfminorversion=7
\usepackage{pdfpages}
\usepackage{transparent}
\newcommand{\incfig}[1]{%
    \def\svgwidth{\columnwidth}
    \import{./figures/}{#1.pdf_tex}
}

\pdfsuppresswarningpagegroup=1

\title{Datastructures and Algorithms: Exercise 2}
\author{David Sermoneta}


\begin{document}
\maketitle
\section*{Exercise 1:}
\begin{lstlisting}
    list = [0]*($\log_{2}(N)$ + 1)
    counter = 0
def convert2binary(N,counter,list):
    1 if N // 2 == 0:
    2   list[counter] = N % 2
    3   return list
    4 else:
    5   list[counter] = N % 2
    6   counter += 1
    7   N = N // 2
    8   return convert2binary(N, counter, list)
\end{lstlisting}
A bit of explaining: the length of our list needs to be $\log_2(N)+1$ long. We get there
by examining that if the binary representation of $N$ is  
\[N = 2^{k} + a_1 2^{k-1} + \ldots + a_{k-1}2^{1} + a_{k}, \text{ where } a_{i}	= 0 \text{ 
 or }1,\] then we can claim that all the terms after $2^{k}$ sum up to less than $2^{k}$ by using
 geometric sums to bound our expression: 
 \begin{align*}
  &N = 2^{k} + m  \leq  2^{k } + \sum\limits_{i=0}^{k-1} 2^{i} = 2^{k} + 2^{k}  - 1  \\
     \implies & m \leq  2^{k} - 1 \leq  2^{k}.
 \end{align*} That is, $N \leq  2^{k+1}$, which finally gives us that \[
 \log_{2}(N) \leq k+1,
 \] which is the length size that our array will need to store each instance of $a_{i}2^{k-i}$. \\
 \noindent Now what remains is the explanation of the actual algorithm. This is just straight 
 forward recursion blended with some euclidean algorithm. The base case covers the case when 
the euclidean algorithm has terminated (that is, when \texttt{N // 2 = 0} and the \texttt{else:}
part of the code adds $N \text{ mod } 2$ (which obviously is either $1$ or $0$), 
and then divides $N$ by $2$ to run the function again, until we get to the base case in 
which case we are done. The counter is just there to make sure we are accessing the right
part of memory.


\section*{Exercise 2:}
We will use the standard rules $\log(a\cdot b) = \log(a) + \log(b)$, $\log(n^{b})=
b\log(n)$ and $\log_{b}(n) = \frac{\log_{c}(n)}{\log_{c}(b)}$.
\begin{proof}[Proof of (a.)]
We note that  \[
\log_{b}(n!) \leq  \log_{b}(n^{n}) = n\log_{b}(n) = O(n\log_{b}(n)) \text{ for all } n\geq 0,
\] and so we are done.
\end{proof}

\begin{proof}[Proof of (b.)]
It suffices to show that $a^{\log_{b}(n)} = n^{\log_{b}(a)}$. We have that \[
LS = a^{\log_{b}(n)}= a^{ \frac{\log_{a}(n)}{\log_{a}(b)}} = n^{\frac{1}{\log_{a}(b)}}
\]  since $\frac{1}{\log_{a}(b)} = \frac{1}{ \frac{\log_{b}(b)}{\log_{b}(a)}}= \log_{b}(a)$, we
have that \[
LS = n^{\log_{b}(a)} = RS.
\] 
\end{proof}

\begin{proof}[Proof of (c.)]
    Suppose $4^{n} \in O(2^{n})$, then for constants $n_0,c$, we'd have that 
\begin{align*}
    & 4^{n} \leq  c 2^{n} \text{ for all } n\geq  n_0 \\
    \iff & 2^{n}2^{n} \leq  c 2 ^{n}  \\
    \iff &  2^{n} \leq  c,
\end{align*} which is clearly a contradiction, since $2^{n}$ is unbounded above. Therefore, we have
disproven that $4^{n} \in O(2^{n})$, and we are done.
\end{proof}

\section*{Exercise 3:}
\begin{proof}[Solution to 3:]
    The arrangement looks like \[
    f_5,f_6,f_7,f_1,f_4,f_2,f_3
    \] Further, the complexity class of each function is
 \begin{align*}
     f_1 \in & O(n^{\frac{3}{2}}), \\
     f_2 \in & O(n^2), \\
     f_3 \in & O(2^{\frac{n}{4}}), \\ 
     f_4 \in & O(n^{3}), \\
     f_5 \in & O(1), \\
     f_6 \in & O(n^{\frac{1}{3}}, \\
     f_7 \in & O(n\log(n))
 \end{align*}
\end{proof}

\section*{Exercise 4:}
\begin{proof}[Proof of (a.)]
Since $T(n)$ is of the form $aT \left( \frac{n}{b} \right) + \Theta(n^{d})$, where $a = 27$, 
$b=3$ and $d= 2$. Since $27 \geq  3^{2}$, we get via the Master theorem that \[
T(n) = \Theta(n^{\log_{3}(27)} = \Theta(n^{3}),
\]  and we are done.
\end{proof}

\begin{proof}[Proof of (b.)]
 Same procedure, set $a = 32$, $b= 2$ and $d= 5$, and we get that \[
 T(n) = aT\left( \frac{n}{b} \right) + \Theta(n^{d}).
 \]   Since $32 = 2^{5},$ Master theorem allows us to conclude that \[
 T(n) = \Theta(n^{5}\log_{2}(n))
 \] 
\end{proof}

\begin{proof}[Proof of (c.)]
   We prove by induction that $T(n) = n!\Theta(1)$. Base case is true, since $T(1) = 
   \Theta(1) = 1!\Theta(1)$.
   Now suppose it were true for some integer $k \geq 2$, then  \[
   T(k+1) =(k+1)T(k) = k+1\cdot k!\Theta(1) \text{ by our induction hypothesis},
   \] and so $T(k+1) = (k+1)! \Theta(1)$, and we are done, the induction step is proven and we 
   have that $T(n) = \Theta(n!)$.
\end{proof}
\newpage
\section*{Exercise 5:}
\begin{proof}[Proof of (a)]
For the program to terminate, we need that either $Q$ or $S$ become empty. 
If $S$ is empty at the time in which the program terminates,
then that means that each element in $S$ can also be found in $Q$, since before
removing an element of $S$ in a given iteration of the \texttt{WHILE} loop, 
we need to check that the same element is in $Q$. And so, in the case that $S$ is empty at
the time the program ends, necessarily $S \subseteq Q$. \bigskip \newline
If $Q$ is empty when the program terminates, then that means that the queue is emptied after a 
finite amount of iterations of the \texttt{WHILE} loop. If originally, $\left|Q\right|= n$,
then that implies that out of all the iterations of the \texttt{WHILE} loop, 
$n$ times of those iterations the \texttt{IF} check will be satisfied, that is, the  $n$
elements of $Q$ will also be found in $S$. Further, since  $S$ remains unchanged, after every
iteration, this gives us that the elements of $Q$ are found in $S$, as the $n$
top-most elements, in some random order. That is, $S$'s final $n$ elements consistute a permutation
of $Q$. So if $Q = [q_1,q_2,q_3 ,\ldots, q_{n}]$, then  \[
S = [s_1,s_2, \ldots, s_{m}, q_1' , q_2', \ldots, q_{n}'], \text{ where } q_{i}' \in Q.
\] To prove that these conditions are sufficient by themselves, we consider the two cases by 
themselves:
Suppose that $S \subseteq Q$, at time $t=0$. Since we are never adding any new elements to $S$ or
$Q$, at any given time $t= t_i$, $S \subseteq Q$ and $\left|S\right| = s_{i} \leq r_{i} = 
\left|Q\right|$. Then we can claim that after at most $r_i$ iterations of the \texttt{WHILE}
loop, a match between $S$ and $Q$ must be found, since after $r_i$ iterations $Q$ at time $t_i$ 
and $Q$ at time $t_i + r_i$ will be the same, meaning no elements of $Q$ have been found in $S$ 
contradicting our assumption. Therefore, in the case that $S \subseteq Q$, $S$ will be emptied of
one element after at most $\left|Q\right|$ iterations, leaving us with $S$ being empty after at 
most 
\begin{equation}
\sum\limits_{k= 0}^{\left|S\right| - 1} \left|Q \right| - k \text{ iterations.}
\end{equation}
And that at a given iteration of the \texttt{WHILE} 
loop at time $t =t_0$,

For the other case, suppose $Q \subset  S$, as well as them being at the top of $S$ as mentioned 
above. Then each iteration of the \texttt{WHILE} loop will leave the top of $S$ unchanged, and
either change the position of the front of $Q$, or find a match, and thus decrease the queue by 
one element. Given the assumption, and with the same reasoning as before, $Q$ will eventually
be emptied, since if we don't find a match between the front of $Q$ and top of $S$ in  
$\left|Q\right|$ iterations, then we've gone through all elements of $Q$ without finding a match,
contradicting our assumptioin. 
\end{proof}
\begin{proof}[Solution of (b.i)]
    If $S=$\texttt{[Theo]} then we'll only need one iteration of the \texttt{WHILE} loop, since
    $S$ will be emptied after the first. 
\end{proof}
\begin{proof}[Solution of (b.ii)]
    If $S=$\texttt{[Theo, Max, Paula, Annemarie, Otto]}, it will require the most amount of 
    iterations, in fact, it will require $15$ iterations, since before removing each element from
    $Q$, we need to run the while loop \texttt{length.}$(Q)$ times (the length of $Q$ before the
    removal of the element). 
\end{proof}
\begin{proof}[Solution of c]
Suppose  $S \subseteq Q$, then as we have shown, the program terminates when $S$ ends up empty, 
which will happen when the  \texttt{IF}  statement runs $\left|S\right|$ times. To make this
the worst case scenario, we want the \texttt{WHILE} loop to iterate as many times as possible 
before each eventual removal. As we have shown earlier, the maximal amount of times we can iterate
the loop before eventually removing an element is $\left|Q\right| - k$ where $k$ is the amount 
of elements we have removed from $S$ so far. So before removing anything, it's just  $\left|Q
\right|$, then $\left|Q\right| - 1$ and so on. Giving us the sum \[
\sum\limits_{k=0}^{\left|S\right|- 1} \left|Q\right| - k = \sum\limits_{k=1}^{\left|S\right|} 
\left|Q\right| - k = \left|S\right|\left|Q\right| - \frac{\left|S\right|}{2}
(\left|S\right| + 1).
\] Since $\left|S\right|, \left|Q\right| \in O(n)$, and using product and sum rules for big-$O$, 
we get that this is all \[
\frac{1}{2}(O(n^2) - O(n)) = O(n^2).
\] In the case that $Q \subseteq S$, and the additional condition on $S$, we get the same sum, but
over $\left|Q\right|$ instead, that is, still $O(n^2)$. 
\end{proof}

\section*{Exercise 6:}
First we note that if we have $k$ people, $2^{k}$ different subsets of people can be created, 
and we can reason that each subset can represent a room being visited by the people that are in the
subset, giving us that we need at most $\left\lfloor \log_{2}(n) \right\rfloor + 1$ people
to be able to identify each room uniquely, in the following way. 
For a given subset of $G= \{p_0,p_1,\ldots,p_{k}\}$, the room that subset visits can be expressed 
in terms of the binary reresentation of the room number, ranging from  $1$ to $n$. That is, if
$R_{i}$ is the room with index $i$, where  $1\leq i \leq n$. Then we know that $i$ can be
uniquely represented in binary using an ordered list of length $k+1$ so that \[
    [i]_{2} = (a_0, a_1,\ldots, a_{k+1}), \text{ where } a_{j} \in \{0,1\} .
\] From here we can say that person  $P_{h}$ is assigned to room $R_{i}$, if $a_{h} = 1$ in the 
binary representation, and so each subset of people is assigned to a unique room, and we are done.
\end{document}
