\documentclass{article}

\usepackage[a4paper, total={6in, 8in}]{geometry} % Page margins
\usepackage[utf8]{inputenc}
\usepackage{amsmath, amssymb, mathtools, amsfonts, amsthm}
\usepackage{wasysym} % Smiley QED
\usepackage{eulervm} % Font
\usepackage{fancyhdr} % Custom headers and footers
\fancyhead[C]{\thepage} % Page numbering for center header
\usepackage{mdframed}
\usepackage{xcolor}

% list environments               
\usepackage{enumerate}            
\usepackage[shortlabels]{enumitem}      

% figure support
\usepackage{import}
\usepackage{xifthen}
\pdfminorversion=7
\usepackage{pdfpages}
\usepackage{transparent}
\newcommand{\incfig}[1]{%
    \def\svgwidth{\columnwidth}
    \import{./figures/}{#1.pdf_tex}
}

\pdfsuppresswarningpagegroup=1

\usepackage{listings}
\lstset{basicstyle = \ttfamily, mathescape}
\title{DSA hand-in 1}
\author{David Sermoneta}


\begin{document}
\maketitle
\section*{Exercise 1}
\begin{lstlisting}
function multiplier(a, b):                                     
  1 y = 0
  2 atimesb = 0
  3 WHILE y < b:
  4     x = 0
  5     WHILE x < a:
  6         atimesb = inc(atimesb)
  7         x = inc(x)
  8     y = inc(y)        
  9 RETURN atimesb
\end{lstlisting}
We need three things, a counter to keep track of how 
many times we've added 1 to \texttt{a} (should be \texttt{a} times),
and a counter to keep
track of how many times to add \texttt{a} to itself (should be \texttt{b} times).

The first counter that achieves this is the 
counter \texttt{y}, as we run the first while loop, we will
add \texttt{1} to \texttt{y} until it stops being smaller than \texttt{b}, then 
we'll have known that we've added \texttt{a} to itself exactly \texttt{b} times.
\newpage
\section*{Exercise 2}
\subsection*{Part a)}
Let $A = ($\texttt{apppe, apple, appl, apap, 
appap}$)$. After each iteration of the \texttt{FOR} loop
we get 
 \begin{lstlisting}
j = 0: (apppe, apple, appl, apap, appap)
    1: (apple, apppe, appl, apap, appap)
    2: (appl, apple, apppe, apap, appap)
    3: (apap, appl, apple, apppe, appap)
    4: (apap, appap, appl, apple, apppe)
\end{lstlisting}

\subsection*{Part b)}
To get the worst case runtime of insertion sort,
we want the \texttt{WHILE} loop to execute as much 
as possible, (as all the other steps have fixed
runtimes after a choice of the input size). To get that, 
we want \texttt{key} $\prec$ \texttt{A[i]} for all 
\texttt{i}  $\geq $ \texttt{0}. This is ensured to be the
case if we let $A$ be in reversed order. That is $A'$ should be 
\begin{lstlisting}
    A' = (apppe, apple, appl, appap, apap).
\end{lstlisting}


\section*{Exercise 3}
\subsection*{Part a)}
We know that $T_{B}(n)$ is in $O(T_{A}(n))$ if and only if 
$T_{A}(n)$ is in $\Omega(T_{B}(n))$, that is, if for some positive constants
$c,n_0$, \[
T_{B}(n) \leq cT_{A}\left( n \right), \text{ for all } n \geq  n_0.
\] To solve this, we write out all terms and rewrite a bit to get \[
\frac{5}{2}n^2 < \frac{cn^2}{10}\log_{10}(n) \iff 25 n^2 
< cn^2\log_{10}(n).
\] Here it is easy to see that $c=1$ and $n_0=10^{25}$ is a choice of
constants that help us satisfy the inequality (since $\log_{10}(10^{25})=25$).
Therefore, $T_{B}(n) \in O(T_{A}\left(n \right) )$, and $T_{A}(n) \in \Omega(T_{B}(n)$. 
\bigskip \newline
To be done, we just need to show that $T_{A}(n)$ is not in $O(T_{B}(n))$. We do so via 
contradiction. Suppose it was true, then for some constants $c,n_0  >0$, we have that for all
$n\geq n_0$, \[
\frac{n^2}{10}\log_{10}(n) \leq \frac{5c}{2}n^2 \iff \log_{10}(n) \leq 25c.
\] However, this is clearly not possible, as for $n\geq 10^{25c}$, the inequality does not hold.
Therefore we are done, $T_{B}$ is the better algorithm for general $n$.


\subsection*{Part b)}
First we note that both runtimes of each algorithm are monotone increasing.
From this it becomes clear that for $n\leq 10^{9}$, algorithm $A$ is the 
recommended one, since \[
    T_{A}(10^{9})=\frac{10^{18}}{10}\log_{10}(10^9)= 10^{17}\cdot 9 < 
    10^{17}\cdot 25 =  \frac{5\cdot 10^{18}}{2} = T_{B}(10^{9}),
\] and we are done.

\section*{Exercise 4}
\subsection*{Part a)}
For \texttt{function(100, 0)}, we get the following values of 
\texttt{n} and  \texttt{N}:
\begin{lstlisting}
n: 100 N: 0
n: 33  N: 1
n: 11  N: 2
n: 3   N: 3
n: 1   N: 4 .
\end{lstlisting}

\subsection*{Part b)}
We note that the runtime of  \texttt{function(n,N} only depends on the input of 
\texttt{n}. Therefore, we can denote the runtime by $T(n)$, and we get that 
the check and increment operations only take $\Theta(1)$ time, and so we get that
\[
T(n) = 2\Theta(1) + T\left(\left\lfloor \frac{n}{3} \right\rfloor\right).
\] Iterating this a few times, we get the following pattern  \[
T(n) = 2N'\Theta(1) + T\left(\left\lfloor \frac{n}{3^{N'}} \right\rfloor\right).
\]
It is clear that the algorithm will terminate at $n=1$, so we want
to find for which  $N'$, $\frac{n}{3^{N'}}\leq  1$, which gives
us that $\log_{3}(n) \leq N'$. Putting $N'=\log_{3}(n)$ in our formula gives us \[
T(n) = 2\log_{3}(n)\Theta(1) + T(1) \in O(\log_{3}(n)).
\] And we are done, we have shown that the best case scenario for 
\texttt{function(n,N)} is $O(\log_{3}(n))$.
\newpage
\section*{Exercise 5}
Let $ S = \{ A_0, A_{1}, \ldots, A_{9} \}$ correspond to the 10 stacks of items,
such that for each $A_{i},$ \[
    A_{i} = \{ a_{i,1}, a_{i,2}, \ldots, a_{i,10}\}.
\] 
Keeping to the constraints of our tools, we want to minimize the amount of times we
need to weigh items. So optimally our algorithm would only need to do a single
weighing. And this is certainly possible, if we take one item from $A_0$, 
2 items from $A_1$ and so on until we get to 10 items from $A_{9}$, we will get
the following equation: \[
\text{Balance} = \sum\limits_{k=1}^{10} (k) + (i + 1), \] where $(i+1)$
stands for the extra grams depending on which stack $A_{i}$ has the
counterfeit items.  Written in pseudo-code, we get 
\begin{lstlisting}
function find_counterfeit(S):
    1 items = []
    2 FOR (i = 0 to 9) DO:
    3    A = S[i]
    4    items.add(A[0,i]
    5 counterfeit weight = weigh(items) - 55 
    6 print("Stack number",{counterfeit weight - 1},contains
    7 all counterfeit items."
    8 RETURN (counterfeit weight - 1)
\end{lstlisting}
So the function returns the the index of the stack $A_{i}$ that has all
the counterfeit items, and we are done.


\end{document}
